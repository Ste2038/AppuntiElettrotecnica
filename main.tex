\documentclass[12pt]{article}

\usepackage{amsmath}
\usepackage{graphicx}
\usepackage{subfiles}
\begin{document}
\author{Stefano Giulianelli}
\title{Appunti Elettrotecnica}
\date{Semestre I, 2022/2023}
\maketitle{}
\tableofcontents{}
\newpage
\section{Leggi}
\subsection{Convenzioni}
\subsubsection{Convenzione degli utilizzatori}
La corrente entra nel positivo ed esce nel negativo.
\subsubsection{Convenzione dei generatori}
La corrente entra nel negativo ed esce nel positivo.
\subsection{Potenza}
$P = V * I$
\subsection{Legge di Ohm}
$V = I*R$, $I = \frac{V}{R}$, $R =\frac{V}{I}$
\subsection{KVL: Legge di Kirchhoff per le tensioni}
In una maglia (percorso chiuso), la somma delle tensioni è uguale a 0.
\subsection{KCL: Legge di Kirchoff per le correnti}
In un nodo, la somma delle correnti entranti sottratta la somma delle correnti uscenti è uguale a 0.
\subsection{Partitore di tensione}
$V_i = V\frac{R_i}{R_{eq}}$
\subsection{Partitore di corrente}
$I_1 = I\frac{R_2\parallel R_3\parallel R_4}{R_1 + R_2\parallel R_3\parallel R_4}$
\subsection{Teorema di Millman}
Due soli nodi devono essere presenti, ogni ramo deve contenere:
\\- un resistore
\\- un generatore di tensione e un resistore
\\- un generatore di corrente
\\Può contenere generatori dipendenti, che per la risoluzioni vengono considerati come indipendenti.
\\$v = \frac{\sum_{k=1}^n G_k v_k + A}{\sum_{k=1}^nG_k} = \frac{v_1G_1+v_2G_2+v_3G_3+...+v_nG_n + A}{G_1+G_2+G_3+...+G_n}$
\section{Componenti}
\subsection{Generatori di tensioni}
Se spento si sostituisce con un cortocircuito.
\subsubsection{... in serie}
$V_{eq} = V_1+V_2+V_3+...+V_n$
\subsubsection{... in parallelo}
Non ha senso avere dei generatori di tensione in parallelo.
\subsection{Generatori di corrente}
Se spento si sostituisce con un circuito aperto.
\subsubsection{... in serie}
Non ha senso avere dei generatori di corrente in serie.
\subsubsection{... in parallelo}
$I_{eq} = I_1+I_2+I_3+...+I_n$
\subsection{Resistori}
\subsubsection{... in serie}
$R_{eq} = R_1 + R_2 + R_3 +...+R_n$
\subsubsection{... in parallelo}
$R_{eq} = \frac{1}{R_1}+\frac{1}{R_2}+\frac{1}{R_3}+...+\frac{1}{R_n}$
\\$R_{eq} = \frac{R_1R_2}{R_1+R_2}$
\subsection{Amplificatore operazionale}
\subsubsection{Amplificatore operazionale invertente}
Identificato dalla presenza di un generatore all'ingresso negativo.
\\$v_{-} = v_{+} = 0$
\subsubsection{Amplificatore operazionale non invertente}
Identificato dalla presenza di un generatore ($V_s$) all'ingresso positivo.
\\$v_{-} = v_{+} = v_s$
\subsubsection{Sommatore}
$v_o = -R_o\left(\frac{v_1}{R_1}+\frac{v_2}{R_2}+\frac{v_3}{R_3}\right)$
\subsubsection{Amplificatore differenziale}
$v_u= \left(1+\frac{R_2}{R_1}\right)\frac{R_4}{R_3+R_4}v_{s2}-\frac{R_2}{R_1}v_{s1}$
\subsubsection{Amplificatori operazionale in cascata}
$\frac{v_o}{v_{in}}=\left(-\frac{R_2}{R_1}\right)\left(-\frac{R_4}{R_3}\right)$

\end{document}